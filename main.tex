\documentclass{ctexart}
\usepackage{tabularx}
\usepackage{multirow}
\usepackage{lipsum}
\usepackage{url}
\renewcommand{\familydefault}{\sfdefault}

\begin{document}

\title{揭棺而起模拟赛}
\author{霸气帅哥, Dan, Liu}

\maketitle

\newpage

\topskip0pt
\vspace*{\fill}
\begin{quote}
  \textit{我即使是死了,钉在棺材里了,也要在墓里,用这腐朽的声带喊出:}\textbf{WYS 这种出题人,就应该拿刀子杀!}\footnote{\url{https://www.zhihu.com/question/55569728/answer/145611430}}
\end{quote}
\vspace*{\fill}

\newpage

\section{loss}
\subsection{背景}
贵校学生节盛况空前,学校派出数千名保安,才勉强清出足够的道路,把所有摊位连接起来。这些摊位生意十分红火,每分钟为国家创造数以百亿计的 GDP。

不巧的是,部分校领导听说了这一盛况,决定到学校观摩一个,保安只得把领导要经过的路线清空,沿线摊位摊主因失去生意而哀嚎不断。

这一声声的哀嚎传到了校门旁综合楼二层敬老院的窗内。窗口的瘦老头陷入沉思:贵校面子工程,让国家损失了数以百亿计的 GDP!他开始记录到底损失了多少 GDP,看看是否还要购入今年的国债。一旁的胖老头叹了口气,说:“MD 这都2017年了,为什么还会有人出裸的数据结构题!”

\subsection{描述}
贵校学生节中有 $n$ 个店铺正在营业,每个店铺会在单位时间创造一定的 GDP。校方勉强清理出 $n-1$ 条道路,每条道路两端各是一个店铺,且不经过其他店铺,保证任意两个店铺之间有且仅有一条路径。

忽然凭空出现 $m$ 个领导,按照各自的路线在店铺间穿梭。每个领导会在某时刻进入学校,在随后某时刻离开学校。在校领导在学校的这段时间中,领导经过路线上的\textbf{所有店铺}将不会获得营收,也就无法创造 GDP。

一个十分有社会责任感的瘦老头希望能够统计因为领导的到来,国家一共损失了多少 GDP。但是考虑到学生节内店铺成千上万,瘦老头无法独自计算出这一损失,所以他真诚地希望恰好路过学校门口的你帮他计算出这一数值。

\subsection{输入格式}
第一行将会有两个正整数, $n$ 与 $m$,分别代表店铺数和领导数。
随后一行将包含 $n$ 个\textbf{实数},第 $i$ 个 $p_i$ 将表示店铺 $i$ 单位时间可以创造的 GDP。
随后的 $n-1$ 行中,第 $i$ 行包含两个正整数 $a_i$ 与 $b_i$,表示店铺 $a_i$ 与 $b_i$ 之间有一条道路。

随后的 $m$ 行中,第 $j$ 行将包含两个正整数, $s_j$ 与 $d_j$,以及两个\textbf{实数}, $in_j$ 与 $out_j$。$s_j$ 与 $d_j$ 表示领导 $j$ 的路径的两端店铺,$jn_j$ 与 $out_j$ 表示领导 $j$ 进入与离开学校的时间。

值得注意的是,领导与领导在学校的时间与路径都可能重叠,并且给出领导的顺序与其路径、在校时间无关。

输入数据中所有实数均保留 5 位 小数。
\subsection{输出格式}
仅一行,包含一个实数,表示损失的 GDP 总量。

这一实数应保留 5 位小数。
\subsection{示例}

\begin{tabularx}{\textwidth}{|X|X|}
  \hline
  \textbf{输入} & \textbf{输出} \\
  \hline
  \begin{texttt}
    3 2\newline
    5579.58004 705.81921 5184.95651\newline
    2 1\newline
    3 1\newline
    4 2 1126.05389 1594.82802\newline
    3 1 3769.33675 6068.50935
  \end{texttt} &
  \begin{texttt}
    output
  \end{texttt} \\
  \hline
\end{tabularx}
\subsection{数据范围}

定义一个测试点可能拥有如下性质:

\begin{enumerate}
  \item \textbf{性质A}: 所有店铺至多连接有两条道路
  \item \textbf{性质B}: 所有领导的访问时间不存在重叠
\end{enumerate}

在下表中,标注有 * 的测试点指这一测试点拥有对应性质。

\bigskip

\begin{table}[ht]
  \centering
  \begin{tabularx}{\textwidth}{|X|X|X|X|X|}
    \hline
    \textbf{测试点} & \textbf{n} & \textbf{m} & \textbf{性质A} & \textbf{性质B} \\\hline
    1 & \multirow{6}{*}{$\le 10$}  & \multirow{3}{*}{$\le 10$} & * & * \\\cline{1-1} \cline{4-5}
    2 &                             &                            & * &   \\\cline{1-1} \cline{4-5}
    3 &                             &                            &   &   \\\cline{1-1} \cline{3-5}
    4 &                             & \multirow{6}{*}{$\le 10^4$}& * & * \\\cline{1-1} \cline{4-5}
    5 &                             &                            & * &   \\\cline{1-1} \cline{4-5}
    6 &                             &                            &   &   \\\cline{1-2} \cline{3-5}
    7 & \multirow{3}{*}{$\le 10^4$} &                            & * & * \\\cline{1-1} \cline{4-5}
    8 &                             &                            & * &   \\\cline{1-1} \cline{4-5}
    9 &                             &                            &   & * \\\hline
    10& $\le 10^5$                  & $\le 10^5$                 &   &   \\\hline
  \end{tabularx}
\end{table}

\bigskip

保证:对于所有数据点:
$ 1 \le a_i, b_i, s_j, d_j \le n $, $ 0 < p_i, in_j, out_j < 100 $。

\end{document}
